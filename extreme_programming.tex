\begin{frame}
\tableofcontents

\end{frame}
    
\subsection{Origines et histoire}
\begin{frame}
\frametitle{\insertsubsection}
\end{frame}

\subsection{Valeurs}
\begin{frame}
\frametitle{\insertsubsection}
\end{frame}

\subsection{Principes généraux}
\begin{frame}
\frametitle{\insertsubsection : Planification}
\begin{itemize}
\item Scénarios : le client dicte les différentes situations, les fonctionnalités. Il utilise son langage, sans rentrer dans des détails techniques. Environ 80 $\pm$ 20.
\item Réunion de planification : les scénarios sont découpés ou regroupés pour correspondre à une itération, puis triés selon la volonté du client. On écrit un plan de publication.
\item Itération : de 1 à 3 semaines de développement maximum. Réunion au début pour repartir le travail, aborder les points techniques. Relase à la fin pour proposer rapidement la nouvelle fonctionnalité et obtenir des feedbacks. Pas d'anticipation sur des projets futurs.
\end{itemize}
\begin{center}
$\longrightarrow$ grande disponibilité du client.
\end{center}
\end{frame}

\begin{frame}
\frametitle{\insertsubsection : Management}
\begin{itemize}
\item Communication : open space pour encourager le travail en collaboration ; table de conférence visible pour que chacun puisse prendre part aux discussions.

\item "Réunions debout" journalières\\ 
%courtes réunions ; présentation de ce qui a été réalisé la veille, du programme du jour et des problèmes qui causent des retards.

\item Un rythme réaliste \\
%: une équipe qui doit travailler des heures supplémentaires sera moins efficace ; ajouter des personnes à un projet déjà en retard n'a pas un impact positif.\\
$\longrightarrow$ Mesure de la vitesse du projet : travail effectué durant une itération.

\item Faire travailler chaque personne sur toutes les parties du projet.

\item Améliorer le système en faisant des réunions rétrospectives.

\end{itemize}

\end{frame}

\begin{frame}
\frametitle{\insertsubsection : Conception}
\begin{itemize}
\item Simplicité du code : testable, compréhensible, explorable, explicable.
\item Pas d'ajout prématuré de fonctionnalités.
\item Pas de généralisation si elle est inutile.
\item Pas d'optimisation si elle n'est pas demandée.
\item Remaniement intensif du code : économie de temps et meilleure qualité. $\rightarrow$ suppression de la redondance, élimination de fonctionnalités obsolètes,...
\end{itemize}
\end{frame}

\begin{frame}
\frametitle{\insertsubsection : Implémentation}
\begin{itemize}
\item Standards : toute l'équipe adopte les mêmes règles de programmation.
\item Binôme : le travail se fait toujours par paires. Relecture permantente et concentration accrue, qui assurent qualité du code.
\item Responsabilité collective : les binômes changent régulièrement. Chacun doit pouvoir travailler sur l'ensemble du code et a ainsi une vue gloable du projet.
\item Intégration continue : chaque modification est rapidement intégrée au code principale après validation. Mais chaque ajout se fait l'un après l'autre, pour assurer la stabilité, sur un ordinateur dédié.
\end{itemize}

\begin{center}
$\longrightarrow$  développement parallèle, publication séquentielle 
\end{center}
\end{frame}

\begin{frame}
\frametitle{\insertsubsection : Validation}

\end{frame}

\subsection{Avantages/Inconvénients}
\begin{frame}
\frametitle{\insertsubsection}
\end{frame}
