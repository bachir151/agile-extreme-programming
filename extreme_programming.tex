\begin{frame}
\tableofcontents

\end{frame}
    
\subsection{Origines et histoire}
\begin{frame}
\frametitle{\insertsubsection}
\end{frame}

\subsection{Valeurs}
\begin{frame}
\frametitle{\insertsubsection}
\end{frame}

\subsection{Principes généraux}
\begin{frame}
\frametitle{\insertsubsection : Planification}
\begin{itemize}
\item Scénarios : le client dicte les différentes situations, les fonctionnalités. Il utilise son langage, sans rentrer dans des détails techniques. Environ 80 $\pm$ 20.
\item Réunion de planification : les scénarios sont découpés ou regroupés pour correspondre à une itération, puis triés selon la volonté du client. On écrit un plan de publication.
\item Itération : de 1 à 3 semaines de développement maximum. Réunion au début pour repartir le travail, aborder les points techniques. Relase à la fin pour proposer rapidement la nouvelle fonctionnalité et obtenir des feedbacks. Pas d'anticipation sur des projets futurs.
\end{itemize}
\begin{center}
$\longrightarrow$ grande disponibilité du client.
\end{center}
\end{frame}

\begin{frame}
\frametitle{\insertsubsection : Management}

\end{frame}

\begin{frame}
\frametitle{\insertsubsection : Conception}

\end{frame}

\begin{frame}
\frametitle{\insertsubsection : Implémentation}
\begin{itemize}
\item Standards : toute l'équipe adopte les mêmes règles de programmation.
\item Binôme : le travail se fait toujours par paires. Relecture permantente et concentration accrue, qui assurent qualité du code.
\item Responsabilité collective : les binômes changent régulièrement. Chacun doit pouvoir travailler sur l'ensemble du code et a ainsi une vue gloable du projet.
\item Intégration continue : chaque modification est rapidement intégrée au code principale après validation. Mais chaque ajout se fait l'un après l'autre, pour assurer la stabilité, sur un ordinateur dédié.
\end{itemize}

\begin{center}
$\longrightarrow$  développement parallèle, publication séquentielle 
\end{center}
\end{frame}

\begin{frame}
\frametitle{\insertsubsection : Validation}

\end{frame}

\subsection{Avantages/Inconvénients}
\begin{frame}
\frametitle{\insertsubsection}
\end{frame}