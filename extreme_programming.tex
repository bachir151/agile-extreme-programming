\begin{frame}
\tableofcontents

\end{frame}
    
\subsection{Origines et histoire}
\begin{frame}
\frametitle{\insertsubsection}
\end{frame}

\subsection{Valeurs}
\begin{frame}
\frametitle{\insertsubsection}
\end{frame}

\subsection{Principes généraux}
\begin{frame}
\frametitle{\insertsubsection : Planification}

\end{frame}

\begin{frame}
\frametitle{\insertsubsection : Management}
\begin{itemize}
\item Communication : open space pour encourager le travail en collaboration ; table de conférence visible pour que chacun puisse prendre part aux discussions.

\item "Réunions debout" journalières\\ 
%courtes réunions ; présentation de ce qui a été réalisé la veille, du programme du jour et des problèmes qui causent des retards.

\item Un rythme réaliste \\
%: une équipe qui doit travailler des heures supplémentaires sera moins efficace ; ajouter des personnes à un projet déjà en retard n'a pas un impact positif.\\
$\longrightarrow$ Mesure de la vitesse du projet : travail effectué durant une itération.

\item Faire travailler chaque personne sur toutes les parties du projet.

\item Améliorer le système en faisant des réunions rétrospectives.

\end{itemize}

\end{frame}

\begin{frame}
\frametitle{\insertsubsection : Conception}
\begin{itemize}
\item Simplicité du code : testable, compréhensible, explorable, explicable.
\item Pas d'ajout prématuré de fonctionnalités.
\item Pas de généralisation si elle est inutile.
\item Pas d'optimisation si elle n'est pas demandée.
\item Remaniement intensif du code : économie de temps et meilleure qualité. $\rightarrow$ suppression de la redondance, élimination de fonctionnalités obsolètes,...
\end{itemize}
\end{frame}

\begin{frame}
\frametitle{\insertsubsection : Implémentation}

\end{frame}

\begin{frame}
\frametitle{\insertsubsection : Validation}
\item Tests automatiques des fonctionnalités demandées par le client(définies par le client).
\item Tests unitaires de non-régression pour chaque portion de code.
$\longrightarrow$ créés en amont de l'implémentation, gérés par un framework dédié, lancés à chaque compilation.
\item $\Rightarrow$ Erreurs facilement détectables ; on peut s'assurer du fonctionnement global du système.

\end{frame}

\subsection{Avantages/Inconvénients}
\begin{frame}
\frametitle{\insertsubsection}
\end{frame}
